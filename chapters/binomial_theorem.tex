\large{\chapter{Binomial Theorem}}
For a binomial expansion $(a+b)^n$, there are $(n+1)$ terms and $(a+b+c)^n$ has $\dfrac{(n+1)(n+2)}{2}$ terms.
\section{Expansion of a binomial expression\newline}
\begin{eqnarray}
	\begin{aligned}
		\begin{split}
			(a+b)^n= & ^nC_0 a^n b^0+^nC_1 a^{n-1} b^1+^nC_2 a^{n-2} b^2+ & \\ & \cdots+^nC_n a^0 b^n\text{ }\forall n \in \mathbb{N} & \\ & =\sum_{i=0}^{n} {^nC_{i}} a^{n-i} b^i\text{ }\forall n \in \mathbb{N}
		\end{split}
	\end{aligned}\\
	\begin{aligned}
		\begin{split}
			(a+b)^n= & a^n b^0+na^{n-1}b+\dfrac{n(n-1)}{2!}a^{n-2}b^2 & \\ & +\cdots+\dfrac{n(n-1)\cdots3\cdot2\cdot1}{n!} a^0 b^n+\cdots\infty \text{ }\forall n \in \mathbb{R}
		\end{split}
	\end{aligned}
\end{eqnarray}

\section{Trinomial Expansion}
For $(a+b+c)^n$:
\begin{equation}
	\begin{split}
		(a+b+c+)^n=\sum \dfrac{n!}{i! j! k!} a^i b^j c^k & \\ & \forall\text{ }(i+j+k)=n\text{; }i,j,k,n \in \mathbb{N}
	\end{split}
\end{equation}

\section{Properties of Coefficients}
\begin{equation}
	\text{Sum of Co-efficients: }C_0+C_1+C_2+\cdots+C_{n-1}+C_n=2^n
\end{equation}
\begin{equation}
	\text{Sum of Odd Co-efficients: }C_0+C_2+C_4+\cdots+C_{2n-3}+C_{2n-1}=2^{n-1}
\end{equation}
\begin{equation}
	C_0-C_1+C_2-\cdots+C_{2n-1}-C_{2n}=0
\end{equation}




\section{Pascal's Rule}
For $1 \leq k \leq n$ and  $k,n \in \mathbb{N}$:
\begin{equation}
	{{n}\choose{k}}+{{n}\choose{k-1}} ={{n+1}\choose{k}}
\end{equation}
%End of Chapter--------------------------------------------------------------------------------------------------------------