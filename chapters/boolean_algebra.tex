\large{\chapter{Boolean Algebra}}
Let $B$ be a set of $a,b,c$ with operations sum $(+)$ and product $(\cdot)$.\newline
Then $B$ is said to belong to the Boolean Structure if the following conditions are satisfied:
\begin{table}[htbp]
	\centering
	\begin{tabular}{l l}
		\toprule
		Property & Name of Property\\
		\midrule
		$a+b \in B$ & \\
		$a \cdot b \in B$ & Closure Property\\

		\midrule[0.05pt]
		$a+b=b+a$ &\\
		$a \cdot b= b \cdot a$ & Associative Law\\

		\midrule[0.05pt]
		$a(b+c) = ab + ac $ & \\
		$a+bc=(a+b)(a+c)$ & Commutative Law\\

		\midrule[0.05pt]
		$\lbrace 0,1 \rbrace \in B$ & \\
		$a+0=a$ & \\
		$a+1=1$ & \\
		$a \cdot 0=0$ & \\
		$a \cdot 1=a$ & Laws of $1$ and $0$\\

		\midrule[0.05pt]
		$a+ab=a$ & \\
		$a(a+b)=a$ & Absorption Law\\

		\midrule[0.1pt]
		$(a+b)'=(a'b')$ & De Morgan's Law\\
		\bottomrule
	\end{tabular}
	\caption{Properties of Boolean Algebraic Structure}
	\label{boolean}
\end{table}


%End of Chapter--------------------------------------------------------------------------------------------