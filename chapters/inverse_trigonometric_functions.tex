\large{\chapter{Inverse Circular Trigonometric Function}}
\section{Definition of Inverse Circular Trigonometric Function}
\subsection{For $\sin x$}
$y=\arcsin x$ iff $x=\sin y$, then:
\begin{enumerate}
	\item $y \in [-\frac{\pi}{2},\frac{\pi}{2}]$
	\item domain of $x \in [-1,1]$
	\item $\sin(\arcsin x)=x,\forall x\in[-1,1]$
	\item $\arcsin(\sin y)=y, \forall y\in [-\frac{\pi}{2},\frac{\pi}{2}]$
	\item $\sin x$ is a strictly increasing in the domain $[-\frac{\pi}{2},\frac{\pi}{2}]$ and one-one.
\end{enumerate}

\subsection{For $\cos x$}
$y=\arccos x$ iff $x=\cos y$, then:
\begin{enumerate}
	\item $y \in [0,\pi]$
	\item domain of $x \in [-1,1]$
	\item $\cos(\arccos x)=x,\forall x\in[-1,1]$
	\item $\arccos(\cos y)=y, \forall y\in [0,\pi]$
	\item $\cos x$ is a strictly decreasing in the domain $[0,\pi]$ and one-one.
\end{enumerate}

\subsection{For $\tan x$}
$y=\arctan x$ iff $x=\tan y$, then:
\begin{enumerate}
	\item $y \in [-\frac{\pi}{2},\frac{\pi}{2}]$
	\item domain of $x \in \mathbb{R}$
	\item $\tan(\arctan x)=x,\forall x\in\mathbb{R}$
	\item $\arctan(\tan y)=y, \forall y\in [-\frac{\pi}{2},\frac{\pi}{2}]$
	\item $\tan x$ is a strictly increasing in the domain $[-\frac{\pi}{2},\frac{\pi}{2}]$ and one-one.
\end{enumerate}

\subsection{For $\cot x$}
$y=\cot^{-1} x$ iff $x=\cot y$, then:
\begin{enumerate}
	\item $y \in (0,\pi)$
	\item domain of $x \in \mathbb{R}$
	\item $\cot(\cot^{-1} x)=x,\forall x\in\mathbb{R}$
	\item $\cot^{-1}(\cot y)=y, \forall y\in (0,\pi)$
	\item $\cot x$ is a strictly decreasing in the domain $(0,\pi)$ and one-one.
\end{enumerate}

\subsubsection{For $\sec x$}
$y=\sec^{-1} x$ iff $x=\sec y$
\begin{enumerate}
	\item $y\in\left\lbrace [0,\frac{\pi}{2})\cup(\frac{\pi}{2},\pi]\right\rbrace$
	\item $x\in\left\lbrace(-\infty,-1]\cup[1,\infty)\right\rbrace$
	\item $\sec(\sec^{-1}x)=x,\forall \lvert x \rvert \geq 1$
	\item $\sec^{-1}(\sec y)=y, \forall y \in \left\lbrace [0,\frac{\pi}{2})\cup(\frac{\pi}{2},\pi]\right\rbrace$
\end{enumerate}

\subsection{For $\csc x$}
$y=\csc^{-1} x$ iff $x=\csc y$
\begin{enumerate}
	\item $y\in\left\lbrace [-\frac{\pi}{2},0)\cup(0,\frac{\pi}{2}]\right\rbrace$
	\item $x\in\left\lbrace(-\infty,-1]\cup[1,\infty)\right\rbrace$
	\item $\csc(\csc^{-1}x)=x,\forall \lvert x \rvert \geq 1$
	\item $\csc^{-1}(\csc y)=y, \forall y \in \left\lbrace [-\frac{\pi}{2},0)\cup(0,\frac{\pi}{2}]\right\rbrace$
\end{enumerate}

\section{Negative Arguments}
\begin{align}
	\arcsin (-x)=-\arcsin x\\
	\arctan (-x)=-\arctan x\\
	\csc^{-1} (-x)=-\csc^{-1} x\\
	\arccos (-x)=\pi-\arccos x\\
	\cot^{-1} (-x)=\pi-\cot^{-1} x\\
	\sec^{-1} (-x)=\pi-\sec^{-1} x
\end{align}

\section{Reciprocal Relations}
\begin{align}
	\csc^{-1} x=\arcsin \dfrac{1}{x}\\
	\sec^{-1} x=\arccos \dfrac{1}{x}\\
	\sec^{-1} x=\begin{cases}
		\arctan \dfrac{1}{x}, x>0\\
		\pi+\arctan \dfrac{1}{x}, x<0
	\end{cases}
\end{align}

\section{I.T.F. Identities}
\begin{align}
	\arcsin x+\arccos x=\dfrac{\pi}{2}, \lvert x \rvert \leq 1\\
	\arctan x+\cot^{-1} x=\dfrac{\pi}{2}, x\in\mathbb{R}\\
	\sec^{-1} x+\csc^{-1} x=\dfrac{\pi}{2}, \lvert x \rvert \geq 1
\end{align}

\section{Sum of Two Angles}
\begin{align}
	\arctan x+\arctan y=\arctan \left(\dfrac{x+y}{1-xy}\right)\\
	\arcsin x+\arcsin y=\arcsin (y\sqrt{1-x^2}+x\sqrt{1-y^2})\\
	\arccos x+\arccos y=\arccos (xy-\sqrt{1-x^2}\sqrt{1-y^2})
\end{align}

\section{Difference of Two Angles}
\begin{align}
	\arctan x-\arctan y=\arctan \left(\dfrac{x-y}{1+xy}\right)\\
	\arcsin x-\arcsin y=\arcsin (x\sqrt{1-y^2}-y\sqrt{1-x^2})\\
	\arccos x-\arccos y=\arccos (xy+\sqrt{1-x^2}\sqrt{1-y^2})
\end{align}

\section{Interconversion of Ratios}
\begin{equation}
	\begin{aligned}
		\begin{split}
			\arcsin x & = & \arccos \sqrt{1-x^2}\\
			& = & \arctan \left(\dfrac{x}{\sqrt{1-x^2}}\right)
		\end{split}
	\end{aligned}
\end{equation}

\begin{equation}
	\begin{aligned}
		\begin{split}
			\arccos x & = & \arcsin \sqrt{1-x^2}\\
			& = & \arctan \left(\dfrac{\sqrt{1-x^2}}{x}\right)
		\end{split}
	\end{aligned}
\end{equation}

\begin{equation}
	\begin{aligned}
		\begin{split}
			2\arctan x & = & \arcsin\left(\dfrac{2x}{1+x^2}\right)\\
			& = & \arccos \left(\dfrac{1-x^2}{1+x^2}\right)\\
			& = & \arctan \left(\dfrac{2x}{1-x^2}\right)
		\end{split}
	\end{aligned}
\end{equation}

\section{Miscellaneous Relations}
\begin{align}
	\cos(\arcsin x)=\sin(\arccos x)=\sqrt{1-x^2}\\
	\arctan x=\frac{\pi}{2}-\arctan\left(\frac{1}{x}\right), x>1
\end{align}
%End of Chapter------------------------------------------------