\large{\chapter{Straight Line}}
A straight line can be defined as:

\begin{align}
	y=mx+c\\
	\dfrac{x}{a}+\dfrac{y}{b}=1\text{, where }a\text{ and }b\text{ are the intercepts at x and y axes respectively}\\
	x\cos\alpha + y\sin\alpha = p\text{ (Normal Form)}\\
	Ax+By+C=0\text{ (General Form)}
\end{align}

\paragraph{Equation of Straight Line Passing Through $(x_0,y_0)$ and Slope $m$\newline}
\begin{equation}
	(y-y_0)=m(x-x_0)
\end{equation}

\paragraph{Distance Between Two Points on a Line\newline}
\begin{align}
	\dfrac{y_1-y_2}{\sin\theta}=\dfrac{x_1-x_2}{\cos\theta}=\gamma\\
	\theta=\tan^{-1}m
\end{align}

\paragraph{Angle Between Two Lines\newline}
For two lines with slopes $m_1, m_2$, the angle between them, $\theta$:
\begin{equation}
	\theta=\arctan\left(\dfrac{m_1-m_2}{1+m_1m_2}\right)
\end{equation}

\paragraph{Distance of a Point from a Line\newline}
Line: $ax+by+c=0$
Point: $(g,h)$
\begin{equation}
	S=\dfrac{ag+bh+c}{\sqrt{a^2+b^2}}
\end{equation}

\paragraph{Angle Bisector of a Line}
For the two lines: $a_1x+b_1y+c_1=0$ and $a_2x+b_2y+c_2=0$, the angle bisector is:
\begin{equation}
	\dfrac{a_1x+b_1y+c_1}{\sqrt{a_1^2+b_1^2}}=\dfrac{a_2x+b_2y+c_2}{\sqrt{a_2^2+b_2^2}}
\end{equation}
If the sign of $c_1$ and $c_2$ is the same, then the equation obtained is the internal bisector.

\paragraph{Equation of a Straight Line Passing through the Intersection of Two Lines\newline}
\begin{equation}
	(a_1x+b_1y+c_1)+k(a_2x+b_2y+c_2)=0\text{ }\forall k\in\mathbb{R}
\end{equation}

\paragraph{Relative Position of Points w.r.t. a Line}
For the points $(x_1,y_1)$ and $(x_2,y_2)$:
\begin{align}
	k_1=ax_1+by_1+c\nonumber\\
	k_2=ax_2+by_2+c\nonumber
\end{align}
If $k_1$ and $k_2$ have the same sign, they are on the same side of a line, otherwise on opposite sides.

\paragraph{Ratio of Division of Line Segment}
For any line, $f(x,y)=0$, the ratio in which it divides $(x_1,y_1)$ and $(x_2,y_2)$ is given by:
\begin{equation}
	r=-\dfrac{f(x_1,y_1)}{f(x_2,y_2)}
\end{equation}
If $\begin{cases}r>0\text{, then division is internal}\\r<0\text{, then division is external}\end{cases}$.
%Enf of Chapter---------------------------------------------------------------