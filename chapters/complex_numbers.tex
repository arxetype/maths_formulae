\chapter{Complex Number}
\section{Basic Formulae}
For $z=x+iy$,
\begin{align}
	&\abs{z} = \sqrt{x^2+y^2}\\
	&\tan \theta = \frac{y}{x}\\
	&\bar{z} = x - iy
\end{align}


\section{Arithmetic Operation of Complex Number}
For two complex numbers $z_1=x_1+iy_1$ and $z_2=x_2+iy_2$:
\begin{align}
	&z_1+z_2=(x_1+x_2)+i(y_1+y_2)\\
	&z_1 \cdot z_2=(x_1x_2-y_1y_2)+i(x_1y_2+x_2y_1)\\
	&\abs{z_1\cdot z_2} = \abs{z_1} \cdot \abs{z_2}\\
	&\frac{z_1}{z_2}=\frac{(x_1x_2+y_1y_2)+i(x_2y_1-x_1y_2)}{{a_2}^2+{b_2}^2}\\
	&\abs*{\frac{z_1}{z_2}} = \frac{\abs{z_1}}{\abs{z_2}}
\end{align}


\section{Euler's Formula}
\begin{equation}
	z=re^{i\theta}\text{, where $r= \abs{z}$, $e^{i\theta}=\cos \theta + i \sin \theta$, and $\theta=\tan^{-1}\frac{y}{x}$}
\end{equation}


\section{Trigonometric Ratios in Complex Form}
\begin{align}
	e^{i\theta}+e^{-i\theta} &= 2\cos\theta\\
	\implies \cos \theta &= \frac{e^{i\theta}+e^{-i\theta}}{2}
\end{align}
\begin{align}
	e^{i\theta}-e^{-i\theta} &= 2\sin\theta\\
	\implies \sin \theta &= \frac{e^{i\theta}-e^{-i\theta}}{2}
\end{align}


\section{De Moivre's Formula}
\begin{equation}
	(\cos\theta + i\sin\theta)^n = \cos(n\theta) + i\sin(n\theta)
\end{equation}


\section{Application of Euler's and De Moivre's Formula}
For $z_1 = r_1 e^{i\theta_1}$ and $z_2 = r_2 e^{i\theta_2}$
\begin{align}
	&z_1\cdot z_2 = r_1r_2 e^{i(\theta_1+\theta_2)}\\
	&\frac{z_1}{z_2}=\frac{r_1}{r_2}e^{i(\theta_1-\theta_2)}
\end{align}


\section{Roots of Unity}
\begin{equation}
	\sqrt[n]{1}=e^{i\frac{2k\pi}{n}},\text{ where } k \in [0,n-1]
\end{equation}


\section{Important Relations of Complex Numbers}
\begin{align}
	\abs{z_1+z_2} \leq \abs{z_1} +\abs{z_2}\\
	\abs{z_1-z_2} \leq  \abs{z_1} + \abs{z_2}\\
	\abs{z_1-z_2} \geq  \abs{z_1} - \abs{z_2}\\
	\abs{z_1+z_2} \geq \abs{\abs{z_1} - \abs{z_2}}\\
	\abs{z_1+z_2}^2 = 2(\abs{z_1}^2 + \abs{z_2}^2) % WTF? CHECK
\end{align}
