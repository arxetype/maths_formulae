\large{\chapter{Complex Number}}
\section{Basic Formulae}
For $z=x+iy$,
\begin{align}
	\lvert z \rvert=\sqrt{x^2+y^2}\\
	\tan \theta=\frac{y}{x}\\
	\bar{z}=x-iy
\end{align}
\section{Arithmetic Operation of Complex Number}
For two complex numbers $z_1=x_1+iy_1$ and $z_2=x_2+iy_2$:
\begin{align}
	z_1+z_2=(x_1+x_2)+i(y_1+y_2)\\
	z_1 z_2=(x_1x_2-y_1y_2)+i(x_1y_2+x_2y_1)\\
	\lvert z_1z_2 \rvert=\lvert z_1 \rvert \lvert z_2 \rvert\\
	\dfrac{z_1}{z_2}=\dfrac{(x_1x_2+y_1y_2)+i(x_2y_1-x_1y_2)}{{a_2}^2+{b_2}^2}\\
	\lvert \dfrac{z_1}{z_2} \rvert=\dfrac{\lvert z_1 \rvert}{\lvert z_2 \rvert}
\end{align}

\section{Euler's Formula}
\begin{align}
	z=re^{i\theta}\text{, where $r=\lvert z \rvert$, $e^{i\theta}=\cos \theta + i \sin \theta$, and $\theta=\tan^{-1}\frac{y}{x}$}
\end{align}

\section{Trigonometric Ratios in Complex Form}
\begin{align}
	e^{i\theta}+e^{-i\theta}=2\cos\theta\\
	\Rightarrow\cos \theta=\dfrac{e^{i\theta}+e^{-i\theta}}{2}\\
	e^{i\theta}-e^{-i\theta}=2\sin\theta\\
	\Rightarrow\sin \theta=\dfrac{e^{i\theta}-e^{-i\theta}}{2}
\end{align}

\section{De Moivre's Formula}
\begin{align}
	(\cos\theta+i\sin\theta)^n=\cos (n\theta)+i\sin(n\theta)
\end{align}

\section{Application of Euler's and De Moivre's Formula}
For $z_1=\lvert r_1 \rvert e^{i\theta_1}$ and $z_2=\lvert r_2 \rvert e^{i\theta_2}$
\begin{align}
	z_1z_2=r_1r_2e^{i(\theta_1+\theta_2)}
	\dfrac{z_1}{z_2}=\dfrac{r_1}{r_2}e^{i(\theta_1-\theta_2)}
\end{align}

\section{Roots of Unity}
\begin{align}
	\sqrt[n]{1}=e^{i\frac{2k\pi}{n}},\text{ where } k \in [0,n-1]
\end{align}

\section{Important Relations of Complex Numbers}
\begin{align}
	\lvert z_1+z_2\rvert \leq \lvert z_1 \rvert+\lvert z_2\rvert\\
	\lvert z_1-z_2\rvert \leq \lvert z_1 \rvert+\lvert z_2\rvert\\
	\lvert z_1-z_2\rvert \geq \lvert z_1 \rvert-\lvert z_2\rvert\\
	\lvert z_1+z_2\rvert \geq \lvert \lvert z_1 \rvert-\lvert z_2\rvert\rvert\\
	\lvert z_1+z_2 \rvert^2=2(\lvert z_1 \rvert^2+\lvert z_2 \rvert^2)
\end{align}
%End of Chapter 2-----------------------------------------------------