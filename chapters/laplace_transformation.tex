\large{\chapter{Laplace Transformations}}
The Laplace Transformation of a function $f(t)$ is defined as:
\begin{equation}
	F(s)=\mathcal{L} \lbrace f(t)\rbrace=\lim_{x\to\infty}\int_0^x e^{-st}f(t) dt
\end{equation}
\section{Basic Transformations}
\begin{table}[htbp]
	\centering
	\begin{tabular}{l c}
		\toprule
		$f(t)$&F(s)\\
		\midrule
		$af(t)+bg(t)$&$aF(s)+bG(s)$\\
		$1$&$\dfrac{1}{s}$\\
		$t$&$\dfrac{1}{s^2}$\\
		$t^n$&$\dfrac{n!}{s^{n+1}}$\\
		$e^{at}$&$\dfrac{1}{s-a}$\\
		$\cos (\omega t)$&$\dfrac{s}{s^2+\omega^2}$\\
		$\sin (\omega t)$&$\dfrac{\omega}{s^2+\omega^2} $\\
		$\sinh at$&$\dfrac{a}{s^2-a^2}$\\
		$\cosh at$&$\dfrac{s}{s^2-a^2}$\\
		\bottomrule
	\end{tabular}
	\caption{Table of Laplace Transformations}
	\label{laplace}
\end{table}

\section{Important Relations}
\begin{align}
	\mathcal{L}\lbrace e^{at} f(t)\rbrace=F(s-a)\\
	\mathcal{L}\lbrace t f(t)\rbrace=-F'(s)\\
	\mathcal{L}\lbrace t^n f(t)\rbrace=(-1)^n F^n(s)\\
	\mathcal{L}\lbrace \dfrac{f(t)}{t} \rbrace=\lim_{x\to\infty} \int_s^x F(u) du\\
	\mathcal{L}\lbrace \dfrac{f(t)}{t^n} \rbrace=\lim_{x\to\infty} \int_1\int_2\cdots{\int_s^x}_n F(u) du\cdots du
\end{align}

\section{Convolution}
For two functions $f(t)$ and $g(t)$ be given such that their Laplace transforms are $F(s)$ and $G(s)$, then:
\begin{equation}
	\mathcal{L}\lbrace f(t) \star g(t) \rbrace=F(s)G(s)
\end{equation}
where $f(t) \star g(t)$ is defined as:
\begin{equation}
	\int_0^t f(u)g(t-u)du
\end{equation}

\section{Laplace Transforms of Differentials}
If the Laplace Transform of $f(t)$ is $F(s)$\footnote{Used in initial value problems}:
\begin{align}
	\mathcal{L}\lbrace f'(t) \rbrace & =sF(s)-y(0)\\
	\mathcal{L}\lbrace f''(t) \rbrace & = s^2 F(s)-[s y(0)+y'(0)]\\
	&\vdots\nonumber\\
	\mathcal{L}\lbrace f^n(t) \rbrace & = s^n F(s)-[\sum_{i=0}^{n-1}s^{n-i}y^i(0)]
\end{align}

%End of Chapter---------------------------------------------------------------------------------