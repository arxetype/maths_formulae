\chapter{Differential Equation}
\section{$1^\text{st}$ Order, $1^\text{st}$ Degree Differential Equation}
For the equation:
\begin{equation}
	\dfrac{dy}{dx}+P(x)y=Q(x)\label{eq1}
\end{equation}
Then an Integral Function (I.F.) is defined as:
\begin{equation}
	I.F.=e^{\int P(x)dx}
\end{equation}
Then the solution of the equation \ref{eq1} is given by:
\begin{equation}
	y(I.F.)=\int Q (I.F.) dx
\end{equation}


\section{$2^\text{nd}$ Order, $1^\text{st}$ Degree Differential Equation}
For the equation:
\begin{equation}
	\dfrac{d^2 y}{dx^2}+a\dfrac{dy}{dx}+by=0\label{eq2}
\end{equation}
\begin{center}
	OR
\end{center}
\begin{equation}
	y''+ay'+by=0
\end{equation}

By substituting $y=e^{\lambda x}$, the equation obtained is:
\begin{align}
	\lambda^2 e^{\lambda x}+\lambda e^{\lambda x}+be^{\lambda x}=0\nonumber\\
	\because e^{\lambda x} \neq 0\nonumber\\
	\Rightarrow \lambda^2+a\lambda+b=0\label{eq3}
\end{align}

If $\alpha$ and $\beta$ are the solutions of the equation \ref{eq3}, then the solution of \ref{eq2} can be:
\begin{enumerate}

	\item If $\alpha = \beta$ and $\alpha,\beta\in\mathbb{R}$:
	\begin{equation}
		y=(c_1+c_2x)e^{\alpha x}
	\end{equation}

	\item If $\alpha \neq \beta$ and $\alpha,\beta\in\mathbb{R}$:
	\begin{equation}
		y=c_1 e^{\alpha x}+c_2 e^{\beta x}
	\end{equation}

	\item If $\lambda = \alpha + i \beta$:
	\begin{equation}
		y=e^{\alpha x}\left[A\cos (\beta x)+B\sin (\beta x) \right]
	\end{equation}
\end{enumerate}


\section{Special Cases of Differential Equation}
\subsection{Definition of Inverse Operator}
The operator $D$ is equivalent to $\dfrac{d}{dx}$. If $Df(x)=X$, then $f(x)=\dfrac{1}{D}X=\int X dx$.

\subsection{Special Cases}
\begin{enumerate}
	\item \begin{equation}f(x)=\dfrac{1}{D-a}X=e^{ax}\int Xe^{-ax}dx \end{equation}

	\item \begin{equation}\dfrac{1}{f(D)}e^{ax}=\begin{cases}
			\dfrac{e^{ax}}{f(a)}, f(a)\neq 0\\
			x\dfrac{e^{ax}}{f'(a)}, f(x)=0 \text{ and } f'(a)\neq 0\\
			x^2\dfrac{e^{ax}}{f''(a)}, f(x)=0 \text{ and } f'(a)= 0
		\end{cases}
	\end{equation}

	\item \begin{equation} \dfrac{1}{f(D)}x^m=[f(D)]^{-1} x^m \end{equation}
	$[f(D)]^{-1}$ is expanded and arranged in terms of ascending powers of $D$ and operated on $x^m$.

	\item \begin{enumerate}
		\item
		\begin{equation}
			\begin{aligned}
				\dfrac{1}{f(D)} \sin (ax) &=& \dfrac{1}{\phi(D^2)} \sin (ax)&\\ &=&\dfrac{1}{\phi(-a^2)} \sin (ax)
			\end{aligned}
		\end{equation}

		\item
		\begin{equation}
			\begin{aligned}
				\dfrac{1}{f(D)} \cos (ax) &=& \dfrac{1}{\phi(D^2)} \cos (ax)&\\ &=&\dfrac{1}{\phi(-a^2)} \cos (ax)
			\end{aligned}
		\end{equation}
	\end{enumerate}

	\item
	\begin{enumerate}
		\item
		\begin{equation}
			\begin{aligned}
				\dfrac{1}{f(D)} \sin (ax) &=& \dfrac{1}{\phi(D^2,D)} \sin (ax)&\\ &=&\dfrac{1}{\phi(-a^2,D)} \sin (ax)
			\end{aligned}
		\end{equation}

		\item
		\begin{equation}
			\begin{aligned}
				\dfrac{1}{f(D)} \cos (ax) &=& \dfrac{1}{\phi(D^2,D)} \cos (ax)&\\ &=&\dfrac{1}{\phi(-a^2,D)} \cos (ax)
			\end{aligned}
		\end{equation}
	\end{enumerate}

	\item \begin{enumerate}
		\item
		\begin{equation}
			\begin{aligned}
				\dfrac{1}{f(D)} \sin (ax) &=& \dfrac{\psi(D)}{\phi(D^2)} \sin (ax)&\\ &=&\dfrac{\psi(D)}{\phi(-a^2)} \sin (ax)
			\end{aligned}
		\end{equation}

		\item
		\begin{equation}
			\begin{aligned}
				\dfrac{1}{f(D)} \cos (ax) &=& \dfrac{\psi(D)}{\phi(D^2)} \cos (ax)&\\ &=&\dfrac{\psi(D)}{\phi(-a^2)} \cos (ax)
			\end{aligned}
		\end{equation}
	\end{enumerate}

	\item
	\begin{enumerate}
		\item
		\begin{equation}
			\dfrac{1}{f(D)} \sin (ax) = x\dfrac{1}{f'(D)} \sin (ax)
		\end{equation}

		\item
		\begin{equation}
			\dfrac{1}{f(D)} \cos (ax) = x\dfrac{1}{f'(D)} \cos (ax)
		\end{equation}
	\end{enumerate}
\end{enumerate}


\section{Method of Variation of Parameters}
If the equation is of the form:
\begin{equation}
	\dfrac{d^2y}{dx^2}+a\dfrac{dy}{dx}+by=f\label{eq4}
\end{equation}
where $a,b,f$ are functions of $x$. The solution for \ref{eq4} is obtained by solving for:
\begin{equation}\label{eq5}
	\dfrac{d^2y}{dx^2}+a\dfrac{dy}{dx}+by=0
\end{equation}
If $y_1$ and $y_2$ are the two independent solution of equation \ref{eq5}.

Then the general solution of the equation is:
\begin{equation}
	y=c_1y_1+c_2y_2
\end{equation}
where $c_1$ and $c_2$ are the constants.

The particular solution of equation \ref{eq5} will be:
\begin{equation}
	y=y_1 \left(\int \dfrac{y_2(-f)}{W}dx\right)+y_2\left(\int \dfrac{y_1 f}{W}dx\right)
\end{equation}
$W$ is the Wronskian, which is defined by:
\begin{equation}
	W=\begin{vmatrix}
		y_1&y_2\\
		y_1'&y_2'
	\end{vmatrix}
\end{equation}


\section{Singular and Ordinary Point}
For a differential equation:
\begin{equation}
	P_0 \dfrac{d^n y}{dx^n}+P_1 \dfrac{d^{n-1}y}{dx^{n-1}}+\cdots+P_{n-1} \dfrac{dy}{dx}+P_n y=R(x)
\end{equation}
where $P_0 \cdots P_n$ are functions of $x$.

If at a point $x=x_0$:
\begin{enumerate}
	\item $P_0(x_0) \neq 0$, $x_0$ is an ordinary point.
	\item $P_0(x_0)=0$, $x_0$ is an singular point:
	\begin{enumerate}
		\item
		\begin{align}
			\lim_{x\to x_0}(x-x_0)P_1(x)=c_1\\
			\lim_{x\to x_0}(x-x_0)^2P_2(x)=c_2\\
		\end{align}
		where $c_1$ and $c_2$ are both finite quantities $x_0$ is a regular singular point.
		\item otherwise it is an irregular singular point.
	\end{enumerate}
\end{enumerate}
