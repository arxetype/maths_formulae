\chapter{Linear Programming Problems}
\section{Basic Feasible Solution}
The standard LPP problem has an objective function and conditions.
\begin{center}
	\begin{align*}
		Z = a_1 x_1 + a_2 x_2 + \cdots + a_n x_n\\
		\text{Subject to:}\\
		a_{11} x_{1} + a_{12} x_{2} + \cdots + a_{1n}x_n \leq b_1\\
		a_{21} x_{1} + a_{22} x_{2} + \cdots + a_{2n}x_n \leq b_2\\
		\vdots\\
		a_{m1} x_{1} + a_{m2} x_{2} + \cdots + a_{mn}x_n \leq b_m
	\end{align*}
\end{center}
For a system with $n$ variables and $m$ conditions, the number of basic solutions are: ${n\choose k}$.
For any $n$ - $m$ system there are $n-m$ non-basic variables (NBV) and $m$ basic variables (BV).\\
For the above system, the basic solutions are obtained by:\\
\begin{table}[ht]
	\centering
	\begin{tabular}{p{0.35\linewidth}  p{0.35\linewidth} p{0.3\linewidth} }
		\toprule
		NBV & BV & BFS\\
		\midrule
		$x_1,x_2,\cdots,x_{n-m} = 0$ & $x_{n-m+1} = c_1, \cdots , x_n = c_n$ & If $x_{n-m+1, \cdots , x_n,} \geq 0$ then it is a basic feasible solution.\\
		\vdots & \vdots & \\
		\bottomrule
	\end{tabular}
\end{table}

\subsection{Adjacent Basic Feasible Solutions}
If two adjacent BFS share $m-1$ BV then they are called adjacent varibales.\\
The optimal solution is always a extreme point. Thus, graphically:

\tikzstyle{rect}=[draw, rectangle, fill=white, text width = 3 cm, text centered, minimum height = 1 cm]
\tikzstyle{elli}=[draw, ellipse, fill=white, text width = 3 cm, text centered, minimum height = 1 cm]
\tikzstyle{circ}=[draw, circle, fill=white, text width = 3 cm, text centered, minimum height = 1 cm]
\tikzstyle{diam}=[draw, diamond, fill=white, text width = 2 cm, text centered]
\tikzstyle{line}=[draw, -latex']
\begin{figure}[h]
	\begin{center}
		\begin{tikzpicture}[node distance = 2 cm, auto]
			\node [rect, rounded corners](step1){Start};
			\node [rect, below of =step1](step2){Plot Feasible Region};
			\node [rect, below of =step2](step3){Find BFS and Z-Value};
			\node [rect, below of =step3](step4){Move to Adjacent BFS};
			\node [diam, below of =step4, node distance = 3 cm](step5){Is Z Value Higher?};
			\node [rect, rounded corners, below of =step5, node distance= 3cm](step6){End};
			\node [rect, right of =step5, node distance= 5cm](step7){Repeat};
			\path [line] (step1)--(step2);
			\path [line] (step2)--(step3);
			\path [line] (step3)--(step4);
			\path [line] (step4)--(step5);
			\path [line] (step5)--node[right]{No}(step6);
			\path [line] (step5)--node[above]{Yes}(step7);
			\path [line] (step7)|-(step4);
		\end{tikzpicture}
	\end{center}
\end{figure}


\section{Simplex Method}