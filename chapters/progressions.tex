\chapter{Progression}
\section{Arithmetic Progression (A.P.)}
An arithmetic sequence is $a, a+n, a+2n, \ldots$ or $t_n = a + (n-1) d$, where $a$ is the first term, $d$ is the common difference, and $n$ is the $n^{th}$-term. % CHECK

An arithmetic series is $a + (a+d) + (a+2d) + \ldots$.

\subsection{Sum of A.P. Series}
\begin{align*}
	S_n &= a + (a+d) + \cdots + (a+\overline{n-2}d) + (a + \overline{n-1}d)\\
	S_n &= (a + \overline{n-1}d) + (a + \overline{n-2}d + \cdots + (a + d) +  a\\
	\implies 2S_n &= n(2a + \overline{n-1}d)\\
	\implies S_n &= \frac{n}{2}(2a + \overline{n-1}d)\numberthis
\end{align*}

\subsection{Important Relation}
If the three terms $a,b,c$ are in A.P., then
\begin{align}
	2b = a + c
\end{align}


\section{Geometric Progression (G.P.)}
An geometric sequence is $a, ar, ar^2, \ldots$ or $t_n = ar^{n-1}$, where $a$ is the first term, $r$ is the common ratio, and $n$ is the $n^{th}$-term. %CHECK

An geometric series is $a+ar+ar^2+...\infty$.

\subsection{The Value of 'r'}
\begin{align}
	r = \frac{t_2}{t_1} = \frac{t_3}{t_2} = \cdots = \frac{t_{n}}{t_{n-1}}
\end{align}

\subsection{Sum of a G.P. Series}
For a definite G.P. series, where there are $n$ terms in the series, the sum of the series is:
\begin{equation}
	S_n = \frac{a\abs{r^n-1}}{\abs{r-1}} %CHECK. No need for abs here.
\end{equation}

For an infinite G.P. series the sum of the series is defined for $r<1$. Sum of such a series is:
\begin{equation}
	S_\infty = \frac{a}{1-r}
\end{equation}

\subsection{Important relations}
If the three terms $a,b,c$ are in G.P., then:
\begin{align}
	b^2 = ac
\end{align}


\section{Harmonic Progression (H.P.)}
If $a, b, c$ are terms of an H.P. then $\frac{1}{a}, \frac{1}{b}, \frac{1}{c}$ are in A.P.
\begin{align}
	\frac{2}{b} &= \frac{1}{a} + \frac{1}{c}\\
	\implies b &= \frac{2ac}{a+c}
\end{align}


\section{Arithmetico-Geometric Progression (A.G.P.)}
Sequence $a, (a+d)r, (a+2d)r^2,\ldots, (a+\overline{n-1}d)r^{n-1}$, where $a$ is first term of A.G.P., $d$ is the common difference, and $r$ is the common ratio.

\subsection{Sum of A.G.P.:}
For an infinite A.G.P. series, the sum is defined for $r<1$:
\begin{equation}
	S_\infty = \frac{a}{1-r} + \frac{dr}{(1-r)^2}
\end{equation}


\section{Special Series}
For $n\in\mathbb{N}$
\begin{align}
	1 + 2 + 3 + \cdots + (n-1) + n &= \frac{n(n-1)}{2}\\
	1^2 + 2^2 + 3^2 + \cdots + (n-1)^2 + n^2 &= \frac{n(n + 1)(2n + 1)}{6}\\
	1^3 + 2^3 + 3^3 + \cdots + (n-1)^3 + n^3 &= \left[\frac{n(n-1)}{2}\right]^2
\end{align}

\subsection{Riemann Zeta Function}
\begin{equation}
	\zeta(s)=\sum_{n=1}^\infty \frac{1}{n^s}
\end{equation}

\subsection{Riemann's Infinite Series as an Integration}
\label{sec:riemann_sum}
\begin{equation}
	\lim_{n\to\infty} \frac{1}{n}\sum_{i=r_1}^{r_2} f\left(\frac{i}{n}\right) = \int_{\flatfrac{r_1}{n}}^{\flatfrac{r_2}{n}} f(x) \dd x
\end{equation}
