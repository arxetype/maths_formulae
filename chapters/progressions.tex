\large{\chapter{Progression}}
\section{Arithmetic Progression (A.P.)}
\paragraph{}An arithmetic sequence is $a,a+n,a+2n,...\infty$ or $t_n=a+(n-1)d$, where $a$ is the first term, $d$ is the common difference, and $n$ is the $n^{th}$-term.
\paragraph{}An arithmetic series is $a+(a+d)+(a+2d)+...\infty$.
\subsection{Sum of A.P. Series}
\begin{align}
	S_n=a+(a+d)+...+(a+\overline{n-2}d)+(a+\overline{n-1}d)\nonumber\\
	S_n=(a+\overline{n-1}d)+(a+\overline{n-2}d+...+(a+d)+ a\nonumber\\
	\Rightarrow 2S_n=n(2a+\overline{n-1}d)\nonumber\\
	\Rightarrow S_n=\dfrac{n}{2}(2a+\overline{n-1}d)
\end{align}
\subsection{Important Relation}
If the three terms $a,b,c$ are in A.P., then
\begin{align}
	2b=a+c
\end{align}
%End of AP-----------------------------------------------------------------

\section{Geometric Progression (G.P.)}
\paragraph{}An geometric sequence is $a,ar,ar^2,...\infty$ or $t_n=ar^{n-1}$, where $a$ is the first term, $r$ is the common ratio, and $n$ is the $n^{th}$-term.
\paragraph{}An geometric series is $a+ar+ar^2+...\infty$.
\subsection{The Value of 'r'}
\begin{align}
	r=\frac{t_2}{t_1}=\frac{t_3}{t_2}=...=\frac{t_{n}}{t_{n-1}}
\end{align}

\subsection{Sum of a G.P.  Series}
\paragraph{}For a definite G.P. series, where there are $n$ terms in the series, the sum of the series is:
\begin{align}
	S_n=\dfrac{a\lvert r^n-1 \rvert}{\lvert r-1 \rvert}
\end{align}

\paragraph{}For an infite G.P. series the sum of the series is defined for $r<1$. Sum of such a series is:
\begin{align}
	S_\infty=\dfrac{a}{1-r}
\end{align}

\subsection{Important relations}
If the three terms $a,b,c$ are in G.P., then:
\begin{align}
	b^2=ac
\end{align}
%End of GP--------------------------------------------------------------------
\section{Harmonic Progression (H.P.)}
If $a,b,c$ are terms of an H.P. then $\frac{1}{a},\frac{1}{b},\frac{1}{c}$ are in A.P.
\begin{align}
	\therefore \dfrac{2}{b}=\dfrac{1}{a}+\dfrac{1}{c}\\
	\Rightarrow b=\dfrac{2ac}{a+c}
\end{align}
%End of HP-------------------------------------------------------------------

\section{\large{Arithmetico-Geometric Progression (A.G.P.)}}
\paragraph{}Sequence $a, (a+d)r, (a+2d)r^2,...,(a+\overline{n-1}d)r^{n-1}$, where $a\rightarrow$first term of A.G.P., $d\rightarrow$common difference, and $r\rightarrow$common ratio.
\subsection{Sum of A.G.P.:}
For an infinite A.G.P. series, the sum is defined for $r<1$:
\begin{align}
	S_\infty=\dfrac{a}{1-r}+\dfrac{dr}{(1-r)^2}
\end{align}
%End of AGP---------------------------------------------------------------------

\section{Special Series}
For $n\in\mathbb{N}$
\begin{align}
	1+2+3+....+(n-1)+n=\dfrac{n(n-1)}{2}\\
	1^2+2^2+3^2+...+(n-1)^2+n^2=\dfrac{n(n+1)(2n+1)}{6}\\
	1^3+2^3+3^3+...+(n-1)^3+n^3=[\dfrac{n(n-1)}{2}]^2
\end{align}
\subsection{Riemann Zeta Function}
\begin{equation}
	\zeta(s)=\sum_{n=1}^\infty \dfrac{1}{n^s}
\end{equation}
\subsection{Riemann's Infinite Series as an Integration}
\label{riemannsum}
\begin{align}
	\lim_{n\to\infty} \dfrac{1}{n}\sum_{i=r_1}^{r_2} f(\frac{i}{n})=\int_{\frac{r_1}{n}}^{\frac{r_2}{n}} f(x) dx
\end{align}
%End of Special Series---------------------------------------------------------
%End of Chapter----------------------------------------------------------------