\chapter{Application of Differentiation}
\section{Rolle's Theorem}
For a function $f(x)$:
\begin{enumerate}
	\item is continuous in $[a,b]$
	\item is differentiable in $(a,b)$
	\item $f(a)=f(b)$,
\end{enumerate}
then there exists a point $x=c$ such that $f'(c)=0$, $c\in(a,b)$


\section{Mean Value Theorem or LaGrange's Theorem}
For a function $f(x)$:
\begin{enumerate}
	\item is continuous in $[a,b]$
	\item is differentiable in $(a,b)$,
\end{enumerate}
then there exists a point $x=c$ such that $f'(c)=\dfrac{f(b)-f(a)}{b-a}$, $c\in(a,b)$, i.e., the tangent is parallel to the line joining the the points $(a,f(a))$ and $(b,f(b))$.


\section{Cauchy's Mean Value Theorem}
For a function $f(x)$ and $g(x)$:
\begin{enumerate}
	\item are continuous in $[a,b]$
	\item are differentiable in $(a,b)$
	\item $g'(x)\neq 0$ in $(a,b)$,
\end{enumerate}
then there exists a point $c\in(a,b)$, such that $\dfrac{f(x)}{g(x)}=\dfrac{f(b)-f(a)}{g(b)-g(a)}$.

\section{Maxima and Minima}
\subsection{Maxima}
For the local maxima of a function $f(x)$:
\begin{enumerate}
	\item $f'(c)=0$ and
	\begin{align}
		\lim_{\epsilon\to c^{-}} f'(\epsilon)>0\nonumber\\
		\lim_{\epsilon\to c^{+}} f'(\epsilon)<0\nonumber
	\end{align}
	\begin{center}
		OR
	\end{center}
	\item $f'(c)=0$ and $f''(x)<0$,
\end{enumerate}
then $f(c)$ is the local maxima point of the function $f(x)$.

\subsection{Minima}
For the local minima of a function $f(x)$:
\begin{enumerate}
	\item $f'(c)=0$ and
	\begin{align}
		\lim_{\epsilon\to c^{-}} f'(\epsilon)<0\nonumber\\
		\lim_{\epsilon\to c^{+}} f'(\epsilon)>0\nonumber
	\end{align}
	\begin{center}
		OR
	\end{center}
	\item $f'(c)=0$ and $f''(x)>0$,
\end{enumerate}
then $f(c)$ is the local minima point of the function $f(x)$.


\section{Taylor's Theorem}
For a function which is differentiable $n$ times:
\begin{equation}
	f(a+h)=f(a)+hf'(a)+\dfrac{h^2}{2!}f''(a)+\cdots+\dfrac{h^{n-1}}{(n-1)!} f^{n-1}(a)+\dfrac{h^n}{x!}R_n
\end{equation}
where $R_n$ is the remainder term.

\subsection{Remainder Term}
\subsubsection{LeGrange's Form}
\begin{equation}
	R_n=f^n (a+\theta h), \theta \in (0,1)
\end{equation}

\subsubsection{Cauchy's Form}
\begin{equation}
	R_n=n(1-\theta)^{n-1}f^n(a+\theta h), \theta \in (0,1)
\end{equation}

\subsection{Conditions for Validity of Expansion}
For validity of Taylor Expansion,the condition
\begin{equation}
	\lim_{n\to\infty} R_n=0
\end{equation}
needs to be satisfied either where $R_n$ is the remainder term in either LeGrange's Form or Cauchy's Form. If the condition is satisfied in a certain domain, then the expansion is valid within that domain only.

\subsection{Taylor's Theorem for Two Variables}
\begin{equation}
	\begin{aligned}
		\begin{split}
			f(a+x,b+y)& = &f(x,y)+\left( a\dfrac{\delta}{\delta x}+b\dfrac{\delta}{\delta y}\right)f (x,y)+&\\
			& &\dfrac{1}{2!}\left( a^2\dfrac{\delta^2}{\delta x^2}+b^2\dfrac{\delta^2}{\delta y^2}\right) f(x,y)+\cdots+&\\
			& &\dfrac{1}{n!}\left( a^n\dfrac{\delta^n}{\delta x^n}+b^n\dfrac{\delta^n}{\delta y^n}\right) f(x+\theta a,y+\theta b), \theta \in (0,1)
		\end{split}
	\end{aligned}
\end{equation}


\section{Maclaurin's Series}
\begin{equation}
	\begin{aligned}
		\begin{split}
			f(x)&=&f(0)+xf'(0)+\dfrac{1}{2!}x^2f''(0)+\dfrac{1}{3!}x^3f'''(0)+\cdots\infty&\\
			&=& \sum_{i=0}^\infty \dfrac{1}{i!} x^i f^i(0)
		\end{split}
	\end{aligned}
\end{equation}

\subsection{Maclaurin's Series with Two Variables}
\begin{equation}
	\begin{aligned}
		\begin{split}
			f(a,b)&=&f(0,0)+\left(a\dfrac{\delta}{\delta x}+b\dfrac{\delta}{\delta x}\right)f(0,0)+&\\& &\dfrac{1}{2!}\left(a^2\dfrac{\delta^2}{\delta x^2}+b^2\dfrac{\delta^2}{\delta x^2}\right)f(0,0)+\cdots\infty&\\
			& = & \sum_{i=0}^\infty \dfrac{1}{n!}\left(a^i\dfrac{\delta^i}{\delta x^i}+b^i\dfrac{\delta^i}{\delta x^i}\right)f(0,0)
		\end{split}
	\end{aligned}
\end{equation}


\section{Curvature}
Curvature is the rate of change of direction w.r.t. arc. Mathematically:
\begin{equation}
	\text{Curvature}=\dfrac{d(\text{direction})}{d(\text{arc})}\nonumber
\end{equation}
\begin{equation}
	\lim_{\Delta s \to 0} \dfrac{\Delta \psi}{\Delta s}=\dfrac{d\psi}{ds}
\end{equation}

\subsection{Radius of Curvature}
\subsubsection{Cartesian Form}
For a curve $y=f(x)$:
\begin{equation}
	\rho=\dfrac{(1+y'^2)^{\frac{3}{2}}}{y''}
\end{equation}
However, this formula fails for $y'\to\infty$.
\subsubsection{Parametric Form}
For a curve defined as $x=\phi(t)$ and $y=\psi(t)$:
\begin{equation}
	\rho=\dfrac{(\ddot{x}^2+\ddot{y}^2)^{\frac{3}{2}}}{x\ddot{y}-y\ddot{x}}
\end{equation}

\subsection{Newton's Formula}
\begin{enumerate}
	\item If the curve passes through origin, and the tangent at origin is the x-axis:
	\begin{equation}
		\rho=\lim_{^{x\to0}_{y\to0}} \dfrac{x^2}{2y}
	\end{equation}
	\item If the curve passes through origin, and the tangent at origin is the y-axis:
	\begin{equation}
		\rho=\lim_{^{x\to0}_{y\to0}} \dfrac{y^2}{2x}
	\end{equation}
	\item If the curve passes through origin and $ax+by+c=0$ is the tangent at origin:
	\begin{equation}
		\rho(0,0)=\dfrac{1}{2}\sqrt{a^2+b^2} \lim_{^{x\to0}_{y\to0}} \dfrac{a^2+y^2}{ax+by}
	\end{equation}
\end{enumerate}

\subsection{Tangent at Origin}
For a curve
\begin{equation}
	\sum c_i x^j y^k=0, i\in\mathbb{N}\text{ and }j,k\in\mathbb{Z}-\lbrace 0 \rbrace
\end{equation}
The curve passes through origin $\because c=0$. Then the lowest degree term equated to $x$ gives the tangent at origin.


\section{Asymptotes}
If the distance between a line $P$ and a curve $f(x)$, $s$ is such that $s\to0$, as $x\to\infty$, then $P$ is the asymptote of $f(x)$. For asymptotes not parallel to x-axis:\newline
Let $y=mx+c$ be the asymptote of the function $y=f(x)$, then:
\begin{align}
	m=\lim_{x\to\infty} \dfrac{y}{x}\\
	c=\lim_{x\to\infty} (y-mx)
\end{align}

\subsection{Asymptote of Algebraic Curves}
For an algebraic curve, passing through origin, defined as:
\begin{equation}
	\begin{aligned}
		\begin{split}
			(a_0x^n+a_1x^{n-1}y^1+a_2x^{n-2}y^2+\cdots+a_{n-1}xy^{n-1}+a_n y^n)+&\\
			(b_0x^{n-1}+b_1x^{n-2}y^1+b_2x^{n-3}y^2+\cdots+b_{n-1}xy^{n-2}+a_n y^{n-1})+&\\
			\cdots=0\nonumber
		\end{split}
	\end{aligned}
\end{equation}
\begin{equation}
	\Rightarrow x^n\phi_n\left(\dfrac{y}{x}\right)+x^{n-1}\phi_{n-1}\left(\dfrac{y}{x}\right)+\cdots+x\phi_1\left(\dfrac{y}{x}\right)=0\nonumber
\end{equation}
The asymptote(s) defined as $y=mx+c$,
\begin{enumerate}
	\item $m$ is the solution for the equation
	\begin{equation}
		\phi_n(m)=0
	\end{equation}
	\item \begin{equation}
		c=-\dfrac{\phi_{n-1}(m)}{\phi_n (m)}
	\end{equation}
	where $c$ is a finite value.
\end{enumerate}
