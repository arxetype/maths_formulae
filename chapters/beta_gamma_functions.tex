\chapter{Beta and Gamma Functions}
For $m,n>0$:
\begin{align}
	\begin{aligned}
		\beta(m,n)&=&\int_0^1 x^{m-1} (1-x)^{n-1} dx&\\
		&=&2\int_0^{\frac{\pi}{2}}\sin^{2m-1}x\cos^{2n-1}x dx
		\begin{split}
		\end{split}
	\end{aligned}
\end{align}

\begin{equation}
	\Gamma(n)=\int_0^\infty e^{-1}x^{n-1} dx
\end{equation}


\section{Important Relations between $\beta(m,n)$ and $\Gamma(n)$ Functions}
\begin{align}
	\Gamma(n)=\dfrac{\Gamma(n+1)}{n}\\
	\Gamma(1)=1\\
	\Gamma\left(\dfrac{1}{2}\right)=\sqrt{\pi}\approx1.772\\
	\Gamma(n+1)=n!, n\in\mathbb{N}\\
	\Gamma(m)\Gamma\left(m+\frac{1}{2}\right)=\sqrt{\pi}\Gamma(2m)\\
	\Gamma(m)\Gamma(m-1)=\pi\csc(m\pi)
\end{align}

\begin{align}
	\beta(m,n)=\beta(n,m)\\
	\beta(m,n)=\dfrac{\Gamma(m)\Gamma(n)}{\Gamma(m+n)}\\
	\beta\left(\dfrac{1}{2},\dfrac{1}{2}\right)=\pi\\
	\int_0^{\frac{\pi}{2}} \sin^p x\cos^q x=\dfrac{1}{2}\dfrac{\Gamma(\frac{p+1}{2})\Gamma(\frac{q+1}{2})}{\Gamma(\frac{p+2}{2})}\\
\end{align}
